\chapter{Stealth Addresses}
Before we go on to explain what stealth technology is and how it is used in
Spectrecoin we need to get some background and some knowledge about how
standard UTXO transactions might be de-anonymised and why stealth technology
came to be used to counter this issue. It will also become apparent why
stealth address technology in itself is not sufficient to provide reasonable
privacy.



The Spectrecoin blockchain is based on the design from Bitcoin Core (except
the consensus mechanisms) and in both systems there are ... possible receiving
addresses. This is an extremely large number and it would give every person on
Earth ... different receiving addresses to use. The fact that it is possible
to re-use an address more than once can be considerd a fluke and is not by
design.



First, let’s have a quick look at how a UTXO blockchain might be deanonymized.
The three major factors that can reduce privacy for the user and are exploitable
through transaction graph analysis are address re-use, change addresses and the
merging of outputs.



Address re-use is treating XSPEC addresses like a bank account where a single
address is used for multiple transactions. XSPEC addresses are not designed to
be used in this way. There are in fact no restrictions on the number of XSPEC
addresses one person can use and for each transaction a new XSPEC address should
be created.



When addresses are re-used, all other transactions performed by that address
can be seen by examining the blockchain. If you are aware of a transaction
made by a person of interest and that transaction comes from the same address
by which this person receives all their payments, then their balances can
easily be determined. You will also be able to look back at the history of
that address, following the chains of transactions, to ascertain what other
information can be extracted.



Address re-use also weakens the security of the coins stored in those addresses.
Transaction signing requires 256 bytes of random data (r-value) so that the
private key cannot be reverse engineered. If the r-value is not truly random
then the private key can be determined, which can be used to sign other
transactions for that particular address. This attack can be negated by
not re-using addresses, as once a transaction is signed from an address,
it remains empty.



Furthermore, each input in a standard transaction must be a full UTXO
from a previous transaction as UTXO’s cannot be partially spent. This
means that if you spend / send less than a full UTXO you will generate
an output that is your change address. Therefore, an attacker examining
the blockchain may generally assume that one output in any transaction
belongs to the creator of the transaction.



Also, if a transaction is generated where two or more UTXOs are pooled
together to create the total input required an assumption can be made
that the addresses merged together belongs to the same person.



Let’s then introduce Stealth Address techniques which allow public keys
appearing in the blockchain to be fully disconnected from “stealth” public
keys which can be publicised by a payee19. The public “stealth” keys
publicised serve as a “master public key” from which “ephemeral public
keys” are derived. The “stealth” public key is never recorded in the
blockchain. This enables the payee to receive infinite un-linkable
payments by publicising only one stealth address. The problem of address
re-use is therefore solved20. A Stealth address therefore is a privacy
technique that protects the privacy of the recipient. The first stealth
address technique was invented by a user known as ‘bytecoin’ in 2011 in
the Bitcointalk forum. Later improvements to stealth tech were proposed
by van Saberhagen in 2013/14 and by Peter Todd in 2014.



The original stealth technology had various problems and on 02/08/2014 one
of the ShadowCash developers known as ‘rynomster’ announced a first fully
working implementation of stealth address technology known as “dual-key
stealth addresses” that solved some of the issues in previous proposals.
A “dual-key stealth address” has two public keys and solves certain problems
associated with previous schemes. For a full and in-depth explanation of
stealth addresses see the paper cited above.



(Courtois N. and Mercer R. (2017). Stealth Address and Key Management
Techniques in Blockchain Systems. In Proceedings of the 3rd International
Conference on Information Systems Security and Privacy ISBN 978-989-
758-209-7, pages 559-566. DOI: 10.5220/0006270005590566)



The use of dual-key stealth addresses provide privacy for the receiver of
funds and introduces various forms of un-linkability, (1) It is hard to
link different public keys/addresses of the same user, (2) It is hard to
link different transactions of the same user, (3) It is hard to link the
sender to the recipient.


\section{Limitations of Stealth Addresses}
As mentioned above, stealth addresses generate a new standard address for
every payment but If you receive for example 10 transactions using your
stealth address you will have 10 UTXOs available to form further
transactions. If you then use some or all of the UTXOs to form a transaction
an observer will be able to link the UTXOs together and assume that they all
belong to one user (merging of outputs). Furthermore, an attacker could
create a number of dust transactions with a stealth address and then monitor
the blockchain to see if the user ever joins those UTXOs together or with
others, in order to make an input to a higher value transaction in the future.
Blockchain analysis can easily link this. This is the reason why any
cryptocurrency that rely on stealth addresses ONLY is not private.



\section{Anonymous Spectrecoin Conversion}
Dual-key Stealth technology is used in the process to create anonymous SPECTRE
by consuming the equivalent value of XSPEC. The creation of SPECTRE involves
the creation of an ATXO with a bundled one-time key-pair that will allow that
ATXO to be ‘spent’ by providing a valid ring signature using your remaining
public key corresponding to the ATXO you previously created and the
corresponding ‘keyimage’21. The fact that an ATXO has been ‘spent’ is only
known to the sender and an observer cannot determine if the ATXO has been
‘spent’.



\section{Keyimage}
The ‘keyImage’ is the result of a cryptographic one-way function derived
from a user’s one-time keypair. The ‘keyimage’ is unique to the ATXO
contributing the value to the new ATXO being created in an anonymous
transaction. The ‘keyimage’ is then recorded in the blockchain to prevent
double spends, but without revealing which ATXO is the value-contributing
member in the ring signature. Although the ‘keyimage’ is recorded in the
blockchain it cannot be reverse engineered due to the one-wayness of the
cryptographic function that generated it. The calculation of the ‘keyimage’
includes the users private key associated with the ATXO being ‘spent’.
Hence, if the user tries to spend the same ATXO again the same ‘keyimage’
will be generated and the system will reject the transaction as that
‘keyimage’ has been seen before.



\begin{lstlisting}
The following SPECTRE denominations are possible: 
1000, 500, 400, 300, 100, 50, 50, 30, 10, 
5, 4, 3, 1, 0.5, 0.4, 0.3, 0.1, 0.0(000000)5, 
0.0(000000)4, 0.0(000000)3, 0.0(000000)1
\end{lstlisting}



We have seen how the use of stealth address tech can be used to solve the
problem of address re-use and to create un-linkable transactions. Now, we
still have the problem of UTXOs being linked together in future transactions.
To resolve this issue Spectrecoin employs the use of ring signatures in
transactions formed by ATXO outputs using SPECTRE.
