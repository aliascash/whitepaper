\section{Keyimage}
The ‘keyImage’ is the result of a cryptographic one-way function derived from a user’s one-time keypair. The ‘keyimage’ is unique to the ATXO contributing the value to the new ATXO being created in an anonymous transaction. The ‘keyimage’ is then recorded in the blockchain to prevent double spends, but without revealing which ATXO is the value-contributing member in the ring signature. Although the ‘keyimage’ is recorded in the blockchain it cannot be reverse engineered due to the one-wayness of the 
cryptographic function that generated it. The calculation of the ‘keyimage’ includes the users private key associated with the ATXO being ‘spent’. Hence, if the user tries to spend the same ATXO again the same ‘keyimage’ will be generated and the system will reject the transaction as that ‘keyimage’ has been seen before. 

 

The following SPECTRE denominations are possible; 1000, 500, 400, 300, 100, 50, 50, 30, 10, 5, 4, 3, 1, 0.5, 0.4, 0.3, 0.1, 0.0(000000)5, 0.0(000000)4, 0.0(000000)3, 0.0(000000)1  

 

We have seen how the use of stealth address tech can be used to solve the problem of address re-use and to create un-linkable transactions. Now, we still have the problem of UTXOs being linked together in future transactions. To resolve this issue Spectrecoin employs the use of ring signatures in transactions formed by ATXO outputs using SPECTRE. 