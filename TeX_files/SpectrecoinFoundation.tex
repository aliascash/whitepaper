\section{The Spectrecoin Foundation}
The Spectrecoin Foundation CIC is a UK registered ‘\textit{Community Interest Company}‘.
This means that we are strictly a \textbf{not-for-profit} endeavour focused
on developing and promoting our software, \textbf{Spectrecoin}. The remit
for the foundation also covers an effort to encourage wider adoption and
use of Spectrecoin.
\\
\\
\noindent
We are subject to UK rules, regulation and company law and the foundation
is also responsible for managing the ‘\textbf{\textit{Spectrecoin Development Fund}}‘
to further the aims of the foundation. We are therefore accountable to
the UK authorities with regards to our not-for-profit status and we will
file financial reports just like any other UK company that will be publicly
available. In this way users of Spectrecoin can be confident that any funds
coming into the foundation are used only for product development and
promotion. As the UK has strict laws and regulation around how companies
operate, we feel that we achieve maximum transparency as opposed to a company
registered in some offshore territory or other tax haven with relaxed but
opaque corporate regulation and taxation.



\subsection{Benefits for Users and Investors}
\begin{itemize}
	\item The ‘\textbf{\textit{Spectrecoin Development Fund}}‘ is managed
	within a regulated corporate structure specifically set up to be a
	not-for-profit organisation where the stated objective is to benefit
	the community. This cannot be changed unless the corporation as such
	is dissolved.
	\item The current members of the foundation are dedicated to the long-term
	development of Spectrecoin. The members have a personal stake in the
	project and a responsibility under UK law to manage the funds and further
	the stated aim of the foundation.
	\item There are currently 4 members (\textit{see below}), but members can
	join and members can leave. The foundation in effect ‘\textit{owns}‘ the
	Spectrecoin official GitHub repo and the GitHub organisation and is
	therefore recognised as the only entity that will issue official
	Spectrecoin software. The foundation will also own all the right to the
	web domains and any other Spectrecoin intellectual property. That means
	that even if the original members would leave the foundation, other
	members could join and have 100\% control over Spectrecoin resources in
	the future. The current members would relinquish any control over any
	Spectrecoin asset upon leaving the foundation.
	\item Should the development fund allow, the foundation members could be
	paid up industry standard salaries that would be subject to scrutiny
	under UK law. In this way, there are strict limits to the level of
	renumeration members could receive.
	\item The entirety of the funds will be used for development of the
	software, that could include hiring people or using contractors. Should
	the development fund be of substantial value beyond development costs
	and salaries, any surplus would be used for research grants or funding
	of research into privacy technology or to further the adoption of
	Spectrecoin.
	\item The fact that the foundation and a corporate structure exists
	does not in any way detract from the objective of developing privacy
	focused blockchain technology and does not in any way impose any
	restrictions on our research and development by the UK government.
	(\textit{also see disclaimer}).
\end{itemize}



\subsection{Funding}
The Spectrecoin blockchain forked on 21/08/2018 @ 2200 hours (GMT) to
introduce ‘\textit{Development Contribution Blocks}’ (DCB) to secure a
minimum amount of funding for the project. One in six staking block rewards
will be designated DCBs and will be sent to the Spectrecoin Foundation
wallet for further distribution. One in 6 block rewards means that the
stake reward from block 1, 6, 12, 18 and so forth are designated DCB
regardless of who owns the wallet that “\textit{won}” and signed the
contribution block. This means that larger wallets will have a higher
probability of donating. It also means that it is possible to
“\textit{win}” two or more stake rewards 6 blocks apart and the owner
of the wallet will contribute 2 stake rewards in a row. We believe that
this will average out and that on the whole larger wallets will contribute
more. This system can be considered a transitional system and we are working
on an improved version of this and we are also exploring ideas around
future funding.
\\
\\
\noindent
This fund will ensure a future for Spectrecoin, will enable us to pay for
certain services, hire contractors and to pay Spectrecoin core team members
in XSPEC/SPECTRE to enable them to work on the project. We believe this
will give us the opportunity to produce better software and will create
value for investors.



\subsection{Members}
The Spectrecoin Foundation have 4 directors at present who are also the members.
These are currently the same persons who comprise the Spectrecoin Core team
and they are responsible for managing and developing Spectrecoin.
They are:



\begin{description}
	\item[Eirik Korsell]  (founder) - https://www.linkedin.com/in/eirik-korsell-37b053173/
	\item[Philip Mueller] (lead developer) - https://www.linkedin.com/in/philip-mueller-82039585/
	\item[Yves Schumann] (developer) - https://www.linkedin.com/in/yves-schumann-02101014b/
	\item[Neil Borum] (community manager) - https://www.linkedin.com/in/neil-borum-bba0b911/
\end{description}



