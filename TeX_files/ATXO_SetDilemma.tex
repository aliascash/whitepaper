\section{The ‘ATXO\_Set’ Dilemma}

We can say that all the ATXOs created in the same transaction are part of an ‘ATXO\_Set’, i.e. they will 
forever be linked to each other as they were created at the same time and it can be assumed that they all 
belong to the same user. There will always be multiple outputs per transaction because of the 
denomination system, because amounts have to be split into discrete values. 



The following dilemma may then occur; the ATXO picking algorithm (as it is now) could select ATXOs from 
an ‘ATXO\_SET‘ as ‘mixins‘ in a ring-signature and this could potentially compromise privacy by making the 
signature more susceptible to deductive analysis. In other words, an observer could be able to determine 
which of the ‘mixins‘ are fake. 



\subsection{Solutions}
New algorithms have been created to negate these issues and strengthen the privacy of the network. This 
way depending on the random pick of the initial ‘mixins‘ and the denominations to fill, there will be a 
random amount of ‘mixins‘ from the same transaction in the final transaction. 



This will also ensure that each output is only used once as a ‘mixin‘ in one of the ring-signatures (VINs). 
Something which was previously only assured by chance. Same ‘mixin‘ in different VINs can only be fake. 
- When the first 9 ‘mixins‘ for an output are chosen, it is completely random, but is ensured that all ‘mixins‘ 
come from different transactions. 
- When ‘mixins‘ for the second ring-signature are chosen, there is a 33% chance that the algorithm tries to 
choose outputs from the transaction chosen as ‘mixins‘ for the first ring-signature. If no outputs can 
picked that way, a new random output is chosen and the corresponding transaction is added as new 
source of ‘mixins‘. 
- Repeat. 



We believe that these issues also exist in Monero and other CryptoNote based systems and may have been 
described first in a paper entitled \textit{“A Traceability Analysis of Monero’s Blockchain”}\footnote{https://eprint.iacr.org/2017/338.pdf} and in particular in 
chapter: 5.2 Heuristic II: Leveraging Output Merging in this paper.