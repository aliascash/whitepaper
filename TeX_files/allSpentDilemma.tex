\section{The ‘All\_Spent’ Dilemma}
A ‘special‘ situation could occur where all the ATXOs available for a certain denomination are spent except 
for your own ATXO to be used in a transaction. We are referring to this situation as the ‘ALL\_SPENT’ 
dilemma and although this appears to be a very low probability situation in V3 it could have dire 
consequences for privacy and compromise a number of ATXOs. Let’s first explain this dilemma in some 
detail: 

 

A valid ring signature (assume ring size 10) needs: (1) An unspent ATXO of a certain denomination to be 
used/spent in the transaction, and (2) Nine (9) ‘mixins’ of the same denomination (spent or unspent). 
These ‘mixins’ (ATXOs) of the same value as the one being spent provides ‘plausible deniability’ with 
regards to the sender. The sender could own any one of the ten ATXOs used in the ring-signature and an 
observer can not determine which one of the 10 ATXOs forming the ring-signature is being spent. This is at 
core of ring-signature privacy and needs to be preserved. 

 

ATXOs of each denomination are “scattered” along the Spectrecoin blockchain and exists in various blocks 
where they were once created and they can all be used as ‘mixins’ in a ring-signature whether they have 
ever been spent or not. Each ATXO is a ‘unique unit of data’ identified by its associated ‘public key’. 
However, only the owner of an ATXO can determine if an ATXO value has been spent or not as this requires 
the corresponding ‘private key’. 



The ATXO picking algorithm for a ring-signature selects 9 ‘mixins’ at random from the available pool of 
ATXOs. In each block there could be a number of ATXOs of different denominations (spent or unspent) that 
could be selected as the 9 ‘mixins’. 

 

Any observer will be able to establish the following: (1) In each valid ring-signature on the blockchain there 
will be one and only one ATXO of a certain denomination being spent. (2) In each valid ring-signature there 
will be nine ‘mixins’ used of the same denomination. (3) The observer will be able to count the number of 
existing ATXOs for each denomination by scanning the blockchain. (4) The observer will be able to count 
the number of ATXOs for each denomination that has been spent by counting the number of valid ring-
signatures where this denomination has been used. 

 

As a result, if the ‘ALL\_SPENT‘ situation occurs, i.e.: 

 

(number\_of\_\_existing\_ATXOs = number\_of\_spent\_ATXOs) 

 

An observer will be able to categorically determine that each of the ATXOs used as a ‘mixin‘ in the ring-
signature has previously been spent. The sender’s privacy has been compromised as the ‘real‘ input ATXO 
can be identified, and we no longer have the ‘plausible deniability’ afforded by the ring-signature. It is 
important to point out the following regarding this issue: 

 

- This only occurs when ALL the ATXOs of a denomination have been spent and a new transaction is created 
using the depleted denomination. 
- The privacy of the transaction spending the last unspent ATXO, although creating an ‘ALL\_SPENT‘ situation 
is not compromised. 
- If the ‘ALL\_SPENT’ occurs, it would only compromise transaction created after the ‘ALL\_SPENT’. All the 
previous transactions would still retain their privacy. 


\subsection{'ALL\_SPENT' illustration}

...

\subsection{Solutions}
A range of measures are being implemented to negate the so called ‘ALL\_SPENT’ dilemma. We realised 
that this is only likely to occur in a situation where there is a very limited supply of ATXOs of a certain 
denomination. 

Therefore, the main approach to solve this problem is to make sure that there is always a sufficient supply 
of ATXOs of varying denominations. We have therefore introduced an advanced ATXO balancer algorithm. 
This algorithm will measure the number of ATXO existing for each denomination and either seek to 
consolidate ATXO values or split ATXO values as part of the staking transaction in PoAS. This will ensure 
that there are always sufficient supply of ATXO to act as ‘mixins‘. 



In addition a new algorithm has been implemented to detect an ‘ALL\_SPENT‘ situation and if this situation 
does occurs the network will ‘remember‘ the block height (block number) and the picking algorithm will no 
longer pick ‘mixins‘ from below that block height. This ensures that users get the full benefit of the privacy 
the ring-signature offers. 
