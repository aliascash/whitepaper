\section{Proof-of-Stake vs. Proof-of-Work}
Spectrecoin uses both PoSv3 and PoAS algorithms to keep the network consensus
and to secure and confirm transactions. Both PoSv3 and PoAS appear to be more
resilient against various attacks that could be instigated against a
Proof-of-Work (PoW) system like Bitcoin, Litecoin and DASH for example.
It is also known that PoW systems are susceptible to so called 51\% attacks
where a sufficiently funded and motivated attacker can “take control” over
the network and generate double spend transactions12. It is more difficult
to attack a PoS system in this way as it would be infeasible to acquire the
majority of Spectrecoin in circulation and doing so would undermine the value
and remove the incentive for the attack in the first place. There are obviously
other attack vectors, such as the recently discovered so called “Fake Stake”
attack against PoSv313. This has since been fixed by the Spectrecoin developers
and Spectrecoin is no longer susceptible to such an attack.

 

It is also well known that large PoW driven networks expend huge amounts of
energy and appears to lead to some level of centralisation of mining power
due to the huge expense involved in mining new blocks. In a recent research
paper entitled “The Bitcoin Mining Network - Trends, Composition, Average
Creation Cost, Electricity Consumption \& Sources” by Christopher Bendiksen
\& Samuel Gibbons of CoinShares Research14, it was found that the Bitcoin
network expends more energy than the whole country of New Zealand.

 

The report calculated that the global Bitcoin mining industry draws 4.7GW of
power every second. Hashing computations for the Proof-of-Work algorithm
consumed 4.3GW, up 0.4GW from the last CoinShares report in November 2018.
Based on these figures, researchers calculated an annual consumption of
41TWh of electricity. That’s roughly 2.2TWh more than New Zealand – a country
of 4.7M people – consumed in 2017, according to the country’s Electricity
Authority15.

 

In comparison, Spectrecoin will run on a standard Raspberry Pi and in addition
to all the privacy features, Spectrecoin is also truly eco-friendly, sustainable
and ‘green technology’. The estimated loose upper bound, annual consumption for
a Raspberry Pi running Spectrecoin is 16.6kWh.

 

41 TWh = 41*1012 Wh = 41,000,000,000,000 Wh (Bitcoin network) 16.6 KWh = 16.6 * 103 Wh = 16,600 Wh * 1000 (nodes) = 16,600,000 Wh (Spectrecoin node) 

 

That means that the whole Bitcoin network consumes almost 2.5 million times
more energy than what an imaginary Spectrecoin network would consume, assuming
1000 Spectrecoin Raspberry Pi nodes.

 


In summary, the PoSv3 / PoAS protocols are both potentially more secure,
immensely more energy efficient and provides for better decentralisation.
It is beyond the scope of this paper to discuss this further and there are
various discussions around the internet if you are interested in the PoW vs.
PoS debate.

 

In the next sections we will explore the different aspects of Spectrecoin in
more detail. We start on the next page with a detailed introduction to the
Spectrecoin privacy features before we go on to discuss Proof-of-Stake v3
(PoSv3) in detail. We move on to explain the privacy features and the new
Proof-of-Anonymous-Stake (PoAS) protocol.