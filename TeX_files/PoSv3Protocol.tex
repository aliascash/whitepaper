\chapter{The Proof-of-Stake (PoSv3) Protocol}
The following section, \underline{in its entirety}, has been taken
directly from, and is based on a blog post by
Qtum\footnote{https://qtum.org/en} developer ‘\textit{Earlz}’
from 27/07/2017 called \textit{“The missing explanation of Proof
of Stake Version 3”}\footnote{http://earlz.net/view/2017/07/27/1904/the-missing-explanation-of-proof-of-stake-version}
and is the best-known source of information on the PoSv3 protocol.
The Spectrecoin PoSv3 implementation adheres completely to this protocol
description for both blocks and transactions and so the explanation from
‘\textit{Earlz}’ can be used for Spectrecoin.



The core concept of Proof-of-Stake (PoS) is very similar to that of
Proof-of-Work (PoW), a lottery. However, the big difference is that,
in PoS, it is not possible to "get more tickets" to the lottery by
simply changing some data in the block and generate a new ticket.
Instead of the "block hash" being the lottery ticket to judge against
a target, PoS invents the notion of a "kernel hash". The ‘kernel hash’
is composed of several pieces of data that are not readily changeable
in the current block. And so, because the miners do not have an easy
way to modify the ‘kernel hash’, they cannot simply iterate through a
large amount of hash values like in PoW. PoS blocks also add many
additional consensus rules in order to realise it's objectives. First,
unlike in PoW, the coinbase transaction (the first transaction in the
block) must be empty and the reward must be 0 tokens. Instead, to reward
stakers, there is a special "staking transaction" which must always be
the 2nd transaction in the block. A ‘staking transaction’ is defined as
any transaction that:


\begin{itemize}
	\item Has at least one valid VIN
	\item It's first VOUT must be an empty script
	\item It's second VOUT must not be empty
\end{itemize}



Furthermore, ‘staking transactions’ must abide by these rules to be valid
in a block:
\begin{itemize}
	\item The second VOUT must be either a pubkey (not pubkeyhash!) script, 
	or an OP\_RETURN script that is unspendable (data-only) but stores data 
	for a public key
	\item The timestamp in the transaction must be equal to the block timestamp
	\item The total output value of a stake transaction must be less than or 
	equal to the total inputs plus the PoS block reward plus the block's 
	total transaction fees. output <= (input + block\_reward + tx\_fees)
	\item The first spent VIN's output must be confirmed by at least 450 
	blocks (in other words, the coins being spent must be at least 450 
	blocks old)
	\item Though more VINs can used and spent in a ‘staking transaction’, 
	the first VIN is the only one used for consensus parameters
\end{itemize}



These rules ensure that the stake transaction is easy to identify and
ensures that it gives enough info to the blockchain to validate the block.
The empty VOUT method is not the only way staking transactions could have
been identified, but this was the original design from Sunny
King\footnote{https://whitepaperdatabase.com/peercoin-ppc-whitepaper/}
and has worked well enough. Now we understand what a ‘staking transaction’
is, and what rules such transaction must abide by.



The next part is to understand the rules for PoSv3 blocks:
\begin{itemize}
	\item Must have exactly 1 staking transaction.
	\item The staking transaction must be the second transaction in the block.
	\item The coinbase transaction must have 0 output value and a single empty VOUT
	\item The block timestamp must have its bottom 4 bits set to 0 (referred 
	to as a "mask" in the source code). This effectively means the blocktime 
	can only be represented in 16 second intervals, decreasing its granularity
	\item The version of the block must be 7
	\item A block's "kernel hash" must meet the weighted difficulty for PoS
	\item The block hash must be signed by the public key in the staking 
	transaction's second VOUT. The signature data is placed in the block 
	(but is not included in the formal block hash)
	\item The signature stored in the block must be "LowS", which means 
	consisting only of a single piece of data and must be as compressed 
	as possible (no extra leading 0s in the data, or other opcodes)
	\item Most other rules for standard PoW blocks apply (valid merkle 
	hash, valid transactions, timestamp is within time drift allowance, etc)
\end{itemize}



There are a lot of details here that we'll cover in a bit. The most important
part that really makes PoS effective lies in the "kernel hash". The kernel
hash is used similar to PoW (if hash meets difficulty, then block is valid).



However, the kernel hash is not directly modifiable in the context of the
current block. We will first cover exactly what goes into these structures
and mechanisms, and later explain why this design is exactly this way, and
what unexpected consequences can come from minor changes to it
