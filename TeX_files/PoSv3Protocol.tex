\section{The Proof-of-Stake (PoSv3) Protocol}
The following section, \underline{in its entirety}, has been taken
directly from, and is based on a blog post by
Qtum\footnote{https://qtum.org/en} developer ‘\textit{Earlz}’
from 27/07/2017 called \textit{“The missing explanation of Proof
of Stake Version 3”}\footnote{http://earlz.net/view/2017/07/27/1904/the-missing-explanation-of-proof-of-stake-version}
and is the best-known source of information on the PoSv3 protocol.
The Spectrecoin PoSv3 implementation adheres completely to this protocol
description for both blocks and transactions and so the explanation from
‘\textit{Earlz}’ can be used for Spectrecoin.



The core concept of Proof-of-Stake (PoS) is very similar to that of
Proof-of-Work (PoW), a lottery. However, the big difference is that,
in PoS, it is not possible to "get more tickets" to the lottery by
simply changing some data in the block and generate a new ticket.
Instead of the "block hash" being the lottery ticket to judge against
a target, PoS invents the notion of a "kernel hash". The ‘kernel hash’
is composed of several pieces of data that are not readily changeable
in the current block. And so, because the miners do not have an easy
way to modify the ‘kernel hash’, they cannot simply iterate through a
large amount of hash values like in PoW. PoS blocks also add many
additional consensus rules in order to realise it's objectives. First,
unlike in PoW, the coinbase transaction (the first transaction in the
block) must be empty and the reward must be 0 tokens. Instead, to reward
stakers, there is a special "staking transaction" which must always be
the 2nd transaction in the block. A ‘staking transaction’ is defined as
any transaction that:


\begin{itemize}
	\item Has at least one valid VIN
	\item It's first VOUT must be an empty script
	\item It's second VOUT must not be empty
\end{itemize}



Furthermore, ‘staking transactions’ must abide by these rules to be valid
in a block:
\begin{itemize}
	\item The second VOUT must be either a pubkey (not pubkeyhash!) script,
	or an OP\_RETURN script that is unspendable (data-only) but stores data
	for a public key
	\item The timestamp in the transaction must be equal to the block timestamp
	\item The total output value of a stake transaction must be less than or
	equal to the total inputs plus the PoS block reward plus the block's
	total transaction fees. output $<=$ (input + block\_reward + tx\_fees)
	\item The first spent VIN's output must be confirmed by at least 450
	blocks (in other words, the coins being spent must be at least 450
	blocks old)
	\item Though more VINs can used and spent in a ‘staking transaction’,
	the first VIN is the only one used for consensus parameters
\end{itemize}



These rules ensure that the stake transaction is easy to identify and
ensures that it gives enough info to the blockchain to validate the block.
The empty VOUT method is not the only way staking transactions could have
been identified, but this was the original design from Sunny
King\footnote{https://whitepaperdatabase.com/peercoin-ppc-whitepaper/}
and has worked well enough. Now we understand what a ‘staking transaction’
is, and what rules such transaction must abide by.



The next part is to understand the rules for PoSv3 blocks:
\begin{itemize}
	\item Must have exactly 1 staking transaction.
	\item The staking transaction must be the second transaction in the block.
	\item The coinbase transaction must have 0 output value and a single empty VOUT
	\item The block timestamp must have its bottom 4 bits set to 0 (referred
	to as a "mask" in the source code). This effectively means the blocktime
	can only be represented in 16 second intervals, decreasing its granularity
	\item The version of the block must be 7
	\item A block's "kernel hash" must meet the weighted difficulty for PoS
	\item The block hash must be signed by the public key in the staking
	transaction's second VOUT. The signature data is placed in the block
	(but is not included in the formal block hash)
	\item The signature stored in the block must be "LowS", which means
	consisting only of a single piece of data and must be as compressed
	as possible (no extra leading 0s in the data, or other opcodes)
	\item Most other rules for standard PoW blocks apply (valid merkle
	hash, valid transactions, timestamp is within time drift allowance, etc)
\end{itemize}



There are a lot of details here that we'll cover in a bit. The most important
part that really makes PoS effective lies in the "kernel hash". The kernel
hash is used similar to PoW (if hash meets difficulty, then block is valid).



However, the kernel hash is not directly modifiable in the context of the
current block. We will first cover exactly what goes into these structures
and mechanisms, and later explain why this design is exactly this way, and
what unexpected consequences can come from minor changes to it



\subsection{The Proof of Stake Kernel Hash}

\underline{The kernel hash specifically consists of the following exact pieces of data (in order):}

\begin{itemize}
	\item Previous block's "stake modifier" (more detail on this later)
	\item Timestamp from "prevout" transaction (the transaction output that
	is spent by the first VIN of the staking transaction)
	\item The hash of the prevout transaction
	\item The output number of the prevout (ie, which output of the
	transaction is spent by the staking transaction)
	\item Current block time, with the bottom 4 bits set to 0 to reduce
	granularity. This is the only thing that changes during staking process
\end{itemize}



\underline{The stake modifier of a block is a hash of exactly:}
\begin{itemize}
	\item The hash of the prevout transaction in PoS blocks, OR the block
	hash in PoW blocks.
	\item The previous block's stake modifier (the genesis block's stake
	modifier is 0)
\end{itemize}



The only way to change the current kernel hash (in order to mine a block),
is thus to either change your "\textit{prevout}", or to change the current
block time.



A single wallet typically contains many UTXOs. The balance of the wallet is
basically the total amount of all the UTXOs that can be spent by the wallet.
This is of course the same in a PoS wallet. This is important though, because
any output can be used for staking. One of these outputs are what can become
the "\textit{prevout}" in a staking transaction to form a valid PoS block.



Finally, there is one more aspect that is changed in the mining process of a
PoS block. The difficulty is weighted against the number of coins in the
staking transaction. The PoS difficulty ends up being twice as easy to achieve
when staking 2 coins, compared to staking just 1 coin.



If this were not the case, then it would encourage creating many tiny UTXOs
for staking, which would bloat the size of the blockchain and ultimately
cause the entire network to require more resources to maintain, as well as
potentially compromise the blockchain's overall security.



So, if we were to show some pseudo-code for finding a valid kernel hash now,
it would look like:



\vspace{5mm} %5mm vertical space

\lstinputlisting{snippets/kernelHashPseudoCode.txt}

\vspace{5mm} %5mm vertical space



This code isn't so easy to understand as our PoW example, so I'll attempt to
explain it in plain English:



Do the following over and over for infinity:
Calculate the blockTime to be the current time minus itself modulus 16
(modulus is like dividing by 16, but then only instead of taking the
result, taking the remainder)
Calculate the posDifficulty as the network difficulty, multiplied by the
number of coins held by the UTXO.
Cycle through each UTXO in the wallet. With each UTXO, calculate a SHA256
hash using the previous block's stake modifier, as well as some data from
the the UTXO, and finally the blockTime. Compare this hash to the
posDifficulty. If the hash is less than the posDifficulty, then the kernel
hash is valid and you can create a new block.
After going through all UTXOs, if no hash produced is less than the
posDifficulty, then wait 16 seconds and do it all over again.



Now that we have found a valid kernel hash using one of the UTXOs we can
spend, we can create a staking transaction. This staking transaction will
have 1 VIN, which spends the UTXO we found that has a valid kernel hash.
It will have (\textit{at least}) 2 VOUTs. The first VOUT will be empty,
identifying to the blockchain that it is a staking transaction. The second
VOUT will either contain an OP\_RETURN data transaction that contains a single
public key, or it will contain a pay-to-pubkey script. The latter is usually
used for simplicity but using a data transaction for this allows for some
advanced use cases (such as a separate block signing machine) without
needlessly cluttering the UTXO set.



Finally, any transactions from the mempool are added to the block. The only
thing left to do now is to create a signature, proving that we have approved
the otherwise valid PoS block. The signature must use the public key that is
encoded (either as pay-pubkey script, or as a data OP\_RETURN script) in the
second VOUT of the staking transaction. The actual data signed in the block
hash. After the signature is applied, the block can be broadcast to the
network. Nodes in the network will then validate the block and if it finds
it valid and there is no better blockchain then it will accept it into its
own blockchain and broadcast the block to all the nodes it has connection to.



It is highly recommended that you read the blog post by
‘\textit{Earlz}’\footnote{http://earlz.net/view/2017/07/27/1904/the-missing-explanation-of-proof-of-stake-version}
and also the associated white papers for
PoS\footnote{https://whitepaperdatabase.com/peercoin-ppc-whitepaper/} and
PoSv2\footnote{https://blackcoin.org/blackcoin-pos-protocol-v2-whitepaper.pdf}
if you want to fully understand how Proof-of-Stake works. PoSv3 as described
here also forms the basis of Proof-of-Anonymous-Stake that we will explore a
bit further on.
