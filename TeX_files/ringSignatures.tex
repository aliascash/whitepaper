\chapter{Ring Signatures}
In a standard UTXO transaction the sender signs the transactions using his/her private key and the signatory can be explicitly determined and identified. In cryptography, a ring signature is a type of digital signature that can be performed by any member of a defined group of users that each have the required keys. A distinctive ring signature is produced through a process that combines the keys of all possible signers and other values and which are then subject to a hash function. 

 

In cryptography a ring signature is a form of a non-interactive zero knowledge proof22. In layman’s terms, what this means is simply that you can prove the correctness of a statement/transaction to a verifier without leaking any additional information by just using a shared common reference string (public key). This system must include cryptographic completeness, soundness and zero-knowledge. 

 

Completeness means that if the statement is correct, then the verifier will always accept. Soundness is a property of such a system that requires that no prover can make the verifier accept a false or incorrect statement. If the statement is incorrect or false, then the verifier will always reject. The last part is zero knowledge. It is not possible to gain any extra information from the proof itself for any malicious verifier except for the correctness of the statement. 

 

This offers a group member a level of anonymity not attainable through generic digital signature schemes. This is a property known as ‘plausible deniability’, or anonymity with respect to an anonymity set. With a ring size of 10 for example there are 10 possible signatories, i.e. 10 public keys and an observer cannot determine which one corresponds to the SPECTRE spent in the transaction. This is only known to the sender. This protects the privacy of the sender. With every transaction using a ring-signature the network ‘transactional entropy’ increases and it becomes increasingly hard to link input/output on the blockchain. 

 

Look at it like this; scattered along the Spectrecoin blockchain are ATXOs of various denominations from 1000 to 0.00000001 SPECTRE. These ATXOs may be spent or unspent but this cannot be determined by an observer. The proof of an ATXO being spent is formed on an ad-hoc basis through the creation of a ‘keyimage’ and there is nothing contained within the data of the ATXO itself or the transaction data written to the blockchain that signifies if it has ever been ‘spent’. In a standard UTXO transaction on the other hand an observer can explicitly determine that an UTXO has indeed been spent to create a new UTXO. 

 

See the illustrations on the next page to visualise the difference between an UTXO and ATXO. 