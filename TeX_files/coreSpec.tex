\section{Core Specification}

\begin{tabular}{ |p{3cm}||p{9cm}|  }
	\hline
	\multicolumn{2}{|c|}{Country List} \\
	\hline
	Country Name     or Area Name& ISO ALPHA 2 Code\\
	\hline
	Genesis block 				& Block \#1 mined on 20/11/2016 (later transition to PoS only)\\
	\hline
	Distribution  				& ICO that raised 16 BTC w/ subsequent distribution in early 2017\\
	\hline
	Ticker 						& XSPEC\\
	\hline
	Initial supply 				& 20,000,000 XSPEC\\
	\hline
	Network outputs (public) 	& XSPEC – public coins\\
	\hline
	Network outputs (private)	& SPECTRE – private coins\\
	\hline
	Consensus (XSPEC) 			& Proof-of-Stake v.3 (PoSv3)\\
	\hline
	Consensus (SPECTRE) 		& Proof-of-Anonymous-Stake (PoAS)\\
	\hline
	Difficulty retarget 		& Every block\\
	\hline
	Target block time 			& 96 seconds\\
	\hline
	Block reward (PoSv3) 		& 2 XSPEC\\
	\hline
	Block reward (PoAS) 		& 3 SPECTRE\\
	\hline
	Coin maturity (confirmations) 	& 450 for stake reward / 10 for SPECTRE / 6 for XSPEC\\
	\hline
	Max supply 					& No max supply (see illustrations on page 8)\\
	\hline
	Inflation 					& Decreasing over time tending to zero\\
	\hline
	Code repository 			& https://github.com/spectrecoin/spectre\\
	\hline
	Supported platforms / OS 	& MS Windows, OSX, Linux, Raspberry Pi\\
	\hline
	Website 					& https://spectreproject.io/\\
	\hline
	Block explorer 				& https://chainz.cryptoid.info/xspec/\\
	\hline
\end{tabular}



It is important to understand that the total ‘outstanding’ amount is the sum
of XSPEC + SPECTRE. On the next page we have projected both a minimum and a
maximum inflation rate and total ‘outstanding’ amount of XSPEC + SPECTRE over
20 years. The real value of the total ‘outstanding’ amount will depend on the
ratio of XSPEC / SPECTRE created by the two different consensus mechanisms
over time. The minimum values represent a scenario where 100\% of the blocks
are created by PoSv3 and the maximum values represent a scenario where 100\%
of the blocks are created by PoAS.